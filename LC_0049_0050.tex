\documentclass[12pt,a4paper]{article}
\usepackage{charter}
\usepackage[latin1]{inputenc}
\usepackage[left=1.50cm, right=1.50cm, top=1.20cm]{geometry}
\usepackage{amsmath}
\usepackage{amsfonts}
\usepackage{amssymb}
\usepackage{graphicx}
\usepackage{textcomp}
\renewcommand{\baselinestretch}{1.5}
\usepackage{listings}
\usepackage{xcolor}
\definecolor{listinggray}{gray}{0.9}
\definecolor{lbcolor}{rgb}{0.9,0.9,0.9}
\usepackage{float}
%\lstset{
%	backgroundcolor=\color{lbcolor},
%	tabsize=4,    
%	%   rulecolor=,
%	language=[GNU]C++,
%	basicstyle=\scriptsize,
%	upquote=true,
%	aboveskip={1.5\baselineskip},
%	columns=fixed,
%	showstringspaces=false,
%	extendedchars=false,
%	breaklines=true,
%	prebreak = \raisebox{0ex}[0ex][0ex]{\ensuremath{\hookleftarrow}},
%	frame=single,
%	numbers=left,
%	showtabs=false,
%	showspaces=false,
%	showstringspaces=false,
%	identifierstyle=\ttfamily,
%	keywordstyle=\color[rgb]{0,0,1},
%	commentstyle=\color[rgb]{0.026,0.112,0.095},
%	stringstyle=\color[rgb]{0.627,0.126,0.941},
%	numberstyle=\color[rgb]{0.205, 0.142, 0.73},
%	%        \lstdefinestyle{C++}{language=C++,style=numbers}?.
%}
%\lstset{
%	backgroundcolor=\color{lbcolor},
%	tabsize=4,
%	language=C++,
%	captionpos=b,
%	tabsize=3,
%	frame=lines,
%	numbers=left,
%	numberstyle=\tiny,
%	numbersep=5pt,
%	breaklines=true,
%	showstringspaces=false,
%	basicstyle=\footnotesize,
%	%  identifierstyle=\color{magenta},
%	keywordstyle=\color[rgb]{0,0,1},
%	commentstyle=\color{Darkgreen},
%	stringstyle=\color{red}
%}

\lstdefinestyle{customc}{
	belowcaptionskip=1\baselineskip,
	aboveskip={1.2\baselineskip},
	breaklines=true,
	frame=lines,
	numbers=left,
	xleftmargin=\parindent,
	language=C++,
	showstringspaces=false,
	basicstyle=\sffamily,%,\ttfamily,
	keywordstyle=\bfseries\color{green!40!black},
	commentstyle=\itshape\color{purple!40!black},
	identifierstyle=\color{blue},
	stringstyle=\color{orange},
}

\lstdefinestyle{customasm}{
	belowcaptionskip=1\baselineskip,
	frame=L,
	xleftmargin=\parindent,
	language=[x86masm]Assembler,
	basicstyle=\footnotesize\ttfamily,
	commentstyle=\itshape\color{purple!40!black},
}

\lstset{escapechar=@,style=customc}

\begin{document}
	\title{LeetCode: 41---50}
	\maketitle
	\section*{Problem 41: First Missing Positive}
	Given an unsorted integer array, find the first missing positive integer.
	For example,
	\par
	Given [1,2,0] return 3,
	and [3,4,-1,1] return 2.
	\noindent
	\par
	Your algorithm should run in $O(n)$ time and uses constant space
	
	\begin{lstlisting}
    int firstMissingPositive(vector<int>& nums) 
    {
    }
	\end{lstlisting}
	
\section*{Problem 42: Trapping Rain Water}
Given n non-negative integers representing an elevation map where the width of each bar is 1, compute how much water it is able to trap after raining.
\par
For example, 
Given [0,1,0,2,1,0,1,3,2,1,2,1], return 6.
\begin{center}
	\includegraphics[width=\linewidth]{rainwatertrap.png}
\end{center}
The above elevation map is represented by array [0,1,0,2,1,0,1,3,2,1,2,1]. In this case, 6 units of rain water (blue section) are being trapped
\begin{lstlisting}
int trap(vector<int>& height) {}
\end{lstlisting}

\section*{Problem 43: Multiply Strings}
Given two numbers represented as strings, return multiplication of the numbers as a string.
\par
Note:
\begin{itemize}
	\item The numbers can be arbitrarily large and are non-negative.
	\item Converting the input string to integer is NOT allowed.
	\item You should NOT use internal library such as BigInteger.
\end{itemize}
\begin{lstlisting}
 string multiply(string num1, string num2) {}
\end{lstlisting}

\section*{Problem 44: Wildcard Matching}
Implement wildcard pattern matching with support for ``?" and ``*".
\par
``?" Matches any single character.
\par
``*" Matches any sequence of characters (including the empty sequence).
\par
The matching should cover the entire input string (not partial).
\par
Some examples:
\begin{lstlisting}
isMatch("aa","a") = false;
isMatch("aa","aa") = true;
isMatch("aaa","aa") = false;
isMatch("aa", "*") = true;
isMatch("aa", "a*") = true;
isMatch("ab", "?*") = true;
isMatch("aab", "c*a*b") = false;
\end{lstlisting}

\section*{Problem 45: Jump Game II}
Given an array of non-negative integers, you are initially positioned at the first index of the array.
\par
Each element in the array represents your maximum jump length at that position.
\par
Your goal is to reach the last index in the minimum number of jumps.
\par
For example:
\par
Given array A = [2,3,1,1,4]
\par
The minimum number of jumps to reach the last index is 2. (Jump 1 step from index 0 to 1, then 3 steps to the last index.)
\par
\textbf{Note:}
You can assume that you can always reach the last index.
\begin{lstlisting}
int jump(vector<int>& nums) {}
\end{lstlisting}

\section*{Problem 46: Permutations}
Given a collection of distinct numbers, return all possible permutations.
\par
For example,
[1,2,3] have the following permutations:
[
[1,2,3],
[1,3,2],
[2,1,3],
[2,3,1],
[3,1,2],
[3,2,1]
]
\begin{lstlisting}
 vector<vector<int>> permute(vector<int>& nums) {}
\end{lstlisting}
\section*{Problem 47: Permutations II}
Given a collection of numbers that might contain duplicates, return all possible unique permutations.
\par
For example,
\par
[1,1,2] have the following unique permutations:
\par
[
[1,1,2],
[1,2,1],
[2,1,1]
]
\begin{lstlisting}
vector<vector<int>> permuteUnique(vector<int>& nums) {}
\end{lstlisting}

\section*{Problem 48: Rotate Image}
You are given an $n \times n$ 2D matrix representing an image. Rotate the image by 90 degrees (clockwise).
\par
Follow up:
\par
Could you do this in-place?
\begin{lstlisting}
void rotate(vector<vector<int>>& matrix) {}
\end{lstlisting}

\section*{Problem 49: Group Anagrams}
Given an array of strings, group anagrams together.
\par
For example, given: [``eat", ``tea", ``tan", ``ate", ``nat", ``bat"], 
\par
Return:
\par
[
\par
[``ate", ``eat",``tea"],
\par
[``nat",``tan"],
\par
[``bat"]
\par
]
\begin{lstlisting}
vector<vector<string>> groupAnagrams(vector<string>& strs) {}
\end{lstlisting}

\section*{Problem 50: Pow}
Implement Pow(x, n)

\begin{lstlisting}
double myPow(double x, int n) {}
\end{lstlisting}
\end{document}